\section{群的概念}
\begin{definition}\label{group}
	设$G$是一个非空集合, “$\cdot$”是$G$上的一个代数运算,即对所偶的$a, b\in G$,有$a\cdot b\in G$. 如果$G$的运算还满足
	\begin{enumerate}
		\item $\forall a, b, c\in G$,有$(a\cdot b)\cdot c = a\cdot (b\cdot c)$
		\item $\exists e$, $\forall a\in G$, 有$e\cdot a = a\cdot e = a$
		\item $\forall a\in G$, $\exists b\in G$,使得$a\cdot b = b\cdot a = e$
	\end{enumerate}
	则称$G$关于运算“$\cdot$”构成一个\textbf{群}(Group), $e$称为群$G$的\textbf{单位元}(unit element)或\textbf{恒等元}(identity), 3中的$b$称为$a$的\textbf{逆元}。容易证明单位元和逆元的唯一性。如果$G$的运算满足交换律,则称$G$为一个\textbf{Abelian群}。群$G$中的元素的个数称为$G$的\textbf{阶}(Order),记为$|G|$,如果$|G|$有限,则称$G$为\textbf{有限群},否则称为\textbf{无限群}。
\end{definition}
\medskip
\begin{example}
	整数集$\mathbb{Z}$关于数的加法构成群,称为整数加群。
\end{example}
\begin{example}
	全体非零有理数集合$\mathbb{Q}^*$,关于数的乘法构成交换群。
\end{example}
\begin{example}
	实数域$\mathbb{R}$上的全体$n$阶方阵$M_n(\mathbb{R})$关于矩阵的加法构成一个交换群。全体$n$阶可逆方阵$GL_n(\mathbb{R})$关于矩阵的乘法构成非交换群。
\end{example}
\begin{example}
	全体$n$次单位根组成的集合
	\begin{equation}
		U_n = \{x\in\mathbb{C}|x^n = 1\} = \{\cos{\frac{2k\pi}{n}}+i\sin {\frac{2k\pi}{n}}|k = 0, 1, 2, \cdots, n-1\}
	\end{equation}
	关于数的乘法构成一个$n$阶交换群。
\end{example}
\begin{example}
	设$m$是大于$1$的正整数,则$\mathbb{Z}_m$关于剩余类的加法构成群,这个群称为$\mathbb{Z}$的\textbf{模$m$剩余类加群}。
\end{example}
\begin{example}
	设$m$是大于$1$的正整数,记
	\begin{equation}
		U(m) = \{\bar{a}\in \mathbb{Z}_m|(a, ) = 1\},
	\end{equation}
	则$U(m)$关于剩余类的乘法构成群。
\end{example}
\begin{note}
	群$(U(m), \cdot)$称为\textbf{$\mathbb{Z}$的模$m$单位群},这显然是一个交换群,当$p$为素数时,常记做$\mathbb{Z}_m^*$,且$|U(m)| = \phi(m)$,其中$\phi$为Euler Totient函数。
\end{note}
\begin{theorem}
	\begin{enumerate}
		\item 群$G$的单位元与逆元唯一;
		\item $\forall a\in G$, $(a^{-1})^{-1} = a$;
		\item $\forall a, b\in G$, 有$(ab)^{-1} = b^{-1}a^{-1}$;
		\item $\forall a, b, c\in G$,若$ab = ac$或$ba=ca$,则$b=c$.
	\end{enumerate}
\end{theorem}

\section{子群}
\begin{definition}\label{subgroup}
	设$G$是一个群, $H$是$G$的一个非空子集。如果$H$关于$G$的运算也构成群,则称$H$为$G$的一个\textbf{子群}(subgroup),记做$H<G$。
\end{definition}
\begin{example}
	对于任意群$G$,$G$本身以及只含有单位元$e$的子集$H = \{e\}$是$G$的子群,称为$G$的\textbf{平凡子群}(trivial subgroup),其他的子群称为\textbf{非平凡子群}(nontrivial subgroup),群$G$不等于它自身的子群称为$G$的\textbf{真子群}(proper subgroup).
\end{example}
\begin{theorem}
	设$G$为群, $H<G$,则
	\begin{enumerate}
		\item 群$G$的单位元$e$是$H$的单位元;
		\item 对于任意$a\in H$, $a$在$G$中的逆元就是$a$在$H$中的逆元。
	\end{enumerate}
\end{theorem}
\begin{theorem}[子群的判别准则之一]
	设$G$为群, $H$是群$G$的\textbf{非空子集},则$H$称为群$G$的子群的充分必要条件是
	\begin{enumerate}
		\item $\forall a, b\in H$,有$ab\in H$;
		\item $\forall a\in H$, 有$a^{-1}\in H$.
	\end{enumerate}
\end{theorem}
\begin{theorem}[子群的判别准则之二(更为常用)]
		设$G$为群, $H$是群$G$的\textbf{非空子集},则$H$称为群$G$的子群的充分必要条件是$\forall a, b\in H$,有$ab^{-1}\in H$.
\end{theorem}
\begin{example}
	$SL_n(\mathbb R) < GL_n(\mathbb R)$(我们之后会证明,这其实是一个正规子群)。
\end{example}
\begin{note}
	$GL_n(\mathbb R)$称为\textbf{一般线性群}, $SL_n(\mathbb R)$称为\textbf{特殊线性群},这两个群在李代数(Lie Algebra)与微分几何(Differential Geometry)中有重要的意义。
\end{note}
\begin{example}
	设$G$为群,记
	\begin{equation}
		C(G) = \{g\in G| gx=xg,\forall x\in G\}
	\end{equation}
	则$C(G)<G$,称为$G$的\textbf{中心}(center).
\end{example}
\begin{note}
	$C(G)$在后面关于有限群分类中有重要作用,详情参见Sylow定理与群的类方程(class equation。
\end{note}
\begin{theorem}
	群$G$的任意两个子群的交集仍为$G$的子群(事实上,任意多个子群的交均为子群,但子群之并不一定是子群)。
\end{theorem}
\begin{definition}\label{cyclic group}
	定义
	\begin{equation}
		<a> :=\{a^r| r\in \mathbb Z\}
	\end{equation}
	是由一个元素$a$生成的群,称为\textbf{循环群}(cyclic group)。
\end{definition}
\begin{note}
	循环群在有限群的分类中也有重要作用,我们稍后会对此加以讨论。
\end{note}
\section{群的同构}
\begin{definition}\label{homomorphism}
	设$G$和$G'$是两个群, $\phi$是$G$到$G'$的\textbf{双射}(bijection),满足
	\begin{equation}
		\phi(a\cdot b) = \phi(a)\cdot\phi(b),\quad \forall a, b\in G,
	\end{equation}
	则称$\phi$为从群$G$到$G'$的一个\textbf{同构映射}(homomorphism),称群$G$与$G'$\textbf{同构}(isomorphism),记做
	\begin{equation*}
		\phi: G\cong G'
	\end{equation*}
	群$G$到自身的同构称为\textbf{自同构}(automorphism)。
\end{definition}
\medskip
一般地,证明两个群同构分为四步
\begin{enumerate}
	\item 构造$G$到$G'$的对应关系$\phi$,并证明$\phi$是一个映射(在商群中要证明$\phi$是\textbf{良定义}(well-defined)的);
	\item 证明$\phi$是单射,即$\forall x, y\in G$,若$\phi(x) = \phi(y)$,则一定有$x=y$;
	\item 证明$\phi$是满射,即$\forall x'\in G'$,存在$x\in G$使得$\phi(x) = x'$;
	\item 证明$\phi$保持运算, 即$\phi(x\cdot y) = \phi(x)\cdot \phi(y)$(注意区分群同构与环同构的差别).
\end{enumerate}
\medskip
\begin{theorem}[群同构的性质]
	设$\phi$是$G$到$G'$的同构映射, $e$和$e'$分别是$G$与$G'$中的单位元, $a\in G$,则有
	\begin{enumerate}
		\item $\phi(e) = e'$;
		\item $\phi(a^{-1}) = (\phi(a)){-1}$;
		\item $\phi$是可逆映射, 且$\phi^{-1}$为$G'$到$G$的同构映射。
	\end{enumerate}
\end{theorem}
\begin{corollary}
	设$G\cong G'$,若$G$是Abelian群,则$G'$也是Abelian群;且$|G| = |G'|$(若为有限群即为元素个数相同,若为无限群则为基数(cardinal number)相同)。
\end{corollary}
\begin{theorem}
	群的同构实际上在所有群构成的集合中定义了一个等价关系(equivalence relationship),即
	\begin{enumerate}
		\item (反身性)$G\cong G$;
		\item (传递性)若$G\cong G'$,$G'\cong G''$,则$G\cong G''$;
		\item (对称性)若$G\cong G'$,则$G' \cong G$.
	\end{enumerate}
	其中$G, G', G''$都是群。
\end{theorem}
\begin{note}
	通过等价关系,实际上给出了所有群所在集合的一个划分,我们可以通过研究一小部分群搞清楚所有群的结构,这一点在有限群的分类中具有极其重要的意义。
\end{note}
\medskip
设$X$是任意集合,令$S_X$是$X$的全体可逆变换构成的集合,定义两个可逆变换的合成
\begin{equation*}
\begin{split}
	\tau\circ \sigma:\quad &X\rightarrow X,\\
	&x \mapsto \tau(\sigma(x)),\quad \forall x\in X
\end{split}
\end{equation*}
仍为$X$的可逆变换。于是$\circ$是$S_X$的代数运算,容易验证$S_X$关于变换的合成构成群(满足群的四条性质)。
\medskip
\begin{theorem}[Cayley定理]
	每一个群都同构与一个变换群。
\end{theorem}
设$G$是群, $a\in G$, 定义$\phi_a$如下
\begin{equation*}
	\begin{split}
		\phi_a(x) = ax, \quad x\in G
	\end{split}
\end{equation*}
称$\phi_a$为一个左乘变换(左平移),全体左乘变换的集合$G_l = \{\phi_a|a\in G\}$,称为$G$的左正则表示(left regular representation)。容易证明$G\cong G_l$。
\begin{note}
	同理我们可以定义右平移与右正则表示,有完全相同的结论成立。	
\end{note}
